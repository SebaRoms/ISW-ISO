\documentclass[]{beamer}
\usepackage[spanish]{babel}
\usepackage[utf8]{inputenc}

%justificado
\usepackage{ragged2e}
\justifying

%tema
\usetheme{Warsaw}
\usecolortheme{default}
\setbeamercovered{transparent}

\title [Ingeniería en Software]{Personal Software Process - ISO/IEC 12207}
\author[Grupo 1]{Oscar León Trureo\\Sebastián Menéndez Sáez\\Claudio Piña Novoa}
\date{\today}
\institute[]{Universidad Tecnol\'ogica Metropolitana}
%\logo{}


%empieza el Documento
\begin{document}


\begin{frame}
	\maketitle
\end{frame}

\begin{frame}
	\frametitle{Índice}
	\tableofcontents[]
\end{frame}

\section{Personal Software Process}
		\subsection{Introducci\'on}
		
			\begin{frame}
				\begin{center}
					\begin{block}{}
						\begin{center}
							{\huge Personal Software Process}
						\end{center}
					\end{block}
				\end{center}
			\end{frame}			
		
			\begin{frame}{Introducci\'on}
				Dentro del mundo del desarrollo en Software, podemos encontrar las metodolog\'ias, las cuales son un conjunto de buenas pr\'acticas, las cuales nos permiten:\\ \pause
				\begin{itemize}
					\item Crear mejores aplicaciones.\pause
					\item Llevar un mejor proceso de desarrollo.\pause
					\item Agilizar el desarrollo en sí.\pause
					\item etc \ldots
				\end{itemize}				
			\end{frame}
			
			\begin{frame}{Introducci\'on}
				Existen una gran cantidad de metodologías, las cuales pueden estar enfocadas al desarrollo en sí, a la gestión, a la calidad, al desarrollador, como también puede ser una mezcla.\\
				\smallskip
				Personal Software Process, es una metodolog\'ia enfocada a la calidad del desarrollo del software a nivel personal, la cual se basa en factores que veremos más adelante.
			\end{frame}						
			
		\subsection{Historia}
			\begin{frame}{Historia}
				\pause
				\begin{itemize}
					\item Creado en el año 1995 por Watt's S. Humphrey en la Universidad de Carnegie Mellon, en Pittsburgh, Pennsylvania.\\ \pause
					\item El primer curso fue imparti\'o en la Universidad de Carnegie Mellon.\\ \pause
					\item fue plasmado en el libro ``A Discipline for SW Engineering'' de Humphrey.\\
				\end{itemize}
			\end{frame}
		
				\begin{frame}
					\begin{quotation}``La calidad del software está dada por la cantidad de procesos usados para desarrollarlo y mantenerlo''.\end{quotation}
			
				\hfill -- \parbox[t]{.9\textwidth}{Watts S. Humphrey,
				\textit{Creador de Personal Software Process}}

				\end{frame}
		
		\subsection{PSP}
			\begin{frame}{PSP}
				\textbf{Personal Software Process}, que en español significa \textbf{Proceso Personal de Software} (PSP), es un conjunto de buenas pr\'acticas las cuales se enfocan al control del tiempo y en la productividad de los Ingenieros en Software, ya sea en la mantencion de sistemas o en tareas de desarrollo.
				\\ \smallskip
				PSP fue diseñado para ayudar a los profesionales del software para que utilicen constantemente prácticas sanas de ingeniería del software, enseñándoles a planificar y dar seguimiento a un trabajo, utilizar un proceso bien definido y medido, a establecer metas mesurables y finalmente a rastrear constantemente para obtener las metas definidas.
			\end{frame}
			
			\begin{frame}{Principios de PSP}
				Personal Software Process, se caracteriza porque es de uso personal, es por esto que se definen principios enfocados a la individualidad:				
				\begin{itemize}
					\smallskip
					\item Cada ingeniero es diferente, para ser más eficiente, debe planificar su trabajo basándose en su experiencia personal.\\ \pause
					\item Cada ingeniero debe utilizar procesos bien definidos y mesurables.\\ \pause
					\item Los ingenieros deben asumir la responsabilidad personal de la calidad de sus productos.\\ \pause
					\item Cuanto antes se detecten y corrijan los errores, menor será el esfuerzo necesario para cumplir la meta.\\ \pause
					\item Es más efectivo prevenir los defectos, que detectarlos y corregirlos.\\
				\end{itemize}
			\end{frame}						
			
			\begin{frame}{PSP}
				\textbf{PSP} no es un est\'andar, es m\'as bien un alternativa que permite mejorar la forma en la que se construye el software, pero con un enfoque ``individual'', por lo que es muy recomendada para los desarrolladores que estén interesados en mejorar en lo que llamamos ``desarrollo individual''.
				\begin{figure}
    				\scalebox{0.2}{\includegraphics{Imagenes/Orga.jpg}} \caption{Niveles de la Organización}
				\end{figure}
			\end{frame}
				
			\begin{frame}{Pasos para la implementación de PSP}
				\begin{enumerate}
					\item Los ingenieros deben ser entrenados por un instructor calificado de PSP. \\ \smallskip \pause
					\item La Capacitacion es sobre grupos o equipos, y seran grupos que asi lo han sido y seguiran siendo. \\ \smallskip \pause
					\item Requiere un fuerte soporte de administración, en este sentido es necesario que los administradores entiendan PSP, saber como apoyarlos y como monitorear sus avances, sin un adecuado monitoreo los ingenieros caerán otra vez en los malos hábitos. \\ \smallskip \pause
					\item Después de ser bien entrenados y bien administrados lo que sigue es optimizar la interacción entre equipos y aquí entraría Team Software Process, el TSP extiende y refina los metodos de CMM y PSP sobre desarrollo y mantenimiento de equipos, y llegar a lo que se le llama un equipo autodirigido. \\ \smallskip
				\end{enumerate}
			\end{frame}
			
			\begin{frame}{Ciclo de Vida de PSP}
				Personal Software Process, tiene un ciclo de vida el cual consta de 5 fases principales, las cuales son fundamentales para el desarrollo de esta metodología: \\ \pause
				\begin{enumerate}		
					\item Planeación \pause
					\item Diseño de Alto Nivel \pause
					\item Revisión de Alto Nivel \pause
					\item Desarrollo Ciclico \pause
					\item Post Mortem
				\end{enumerate}
			\end{frame}											
				
			\begin{frame}{Fase de Planeación}
				\begin{block}{Entrada}
					Descripción del problema, resumen del proyecto, resumen cíclico, tamaño estimado, tiempo estimado, formas de planeación.
				\end{block}
				\begin{block}{Actividad}
					Requerimientos, tamaño estimado, desarrollo estrategia, estimados de recursos, planificación y programas de tareas, estimación de defectos.
				\end{block}
				\begin{block}{Salida}
					Diseño conceptual, resumen plan, resumen del ciclo, patrones de estimados de tamaño y planeación de tareas, programas de patrones de planeación, registro de tiempos.
				\end{block}
			\end{frame}
			
			\begin{frame}{Fase de Diseño de Producto}
				\begin{block}{Entrada}
					Tipificación requerimientos, diseño conceptual, patrones de estimaciones de tamaño, resumen parte ciclico, seguimiento.
				\end{block}
				\begin{block}{Actividad}
					Especificaciones externas, diseño modular, prototipos, estrategia de desarrollo y documentación, seguimiento.
				\end{block}
				\begin{block}{Salida}
					Diseño de programa, escenarios operacionales, especificación de funciones y lógica, resumen cíclico, seguimiento y estrategias de pruebas.
				\end{block}
			\end{frame}

			\begin{frame}{Fase Revisión o Validación del Diseño}
				\begin{block}{Entrada}
					Programa de diseño, escenarios operacionales, especificación de funciones y lógica, resumen ciclico, seguimiento y estrategia de pruebas y ciclo.
				\end{block}
				\begin{block}{Actividad}
					Diseño de apariencia, verificación de máquinas y lógica, consistencia del diseño, reuso, estrategia de verificación, detectar errores.
				\end{block}
				\begin{block}{Salida}
					Diseño de alto nivel, registro de seguimiento, tiempos y defectos.
				\end{block}
			\end{frame}
			
			\begin{frame}{Fase de Desarrollo o Implementación}
				\begin{block}{Entrada}
					Diseño de alto nivel, registro de seguimiento, tiempos y defectos, ciclo de desarrollo, estrategia de pruebas, patrones de operación y función.
				\end{block}
				\begin{block}{Actividad}
					Diseño de módulos, revisión de diseño, código, revisión de código, compilación, pruebas, aseguramiento de calidad y del ciclo.
				\end{block}
				\begin{block}{Salida}
					Modulos de sw, patrón de diseño, lista de verificación de código y diseño, resumen del ciclo, patrón de reporte de pruebas, registro de tiempo, defectos y seguimiento.
				\end{block}
			\end{frame}			
			
			\begin{frame}{Fase PostMortem, Evaluación Ciclo}
				\begin{block}{Entrada}
					Definición de problema y requerimientos, plan de proyecto y de ciclo, producto de software, patrón de diseño, lista de verificación de código y diseño, resumen del ciclo, patrón de reporte de pruebas, registro de tiempo, defectos y seguimiento.
				\end{block}
				\begin{block}{Actividad}
					Defectos previstos, removidos, tamaño, tiempo del producto.
				\end{block}
				\begin{block}{Salida}
					Producto, listas de verificación, plan de proyecto y ciclo, patrón de reporte de pruebas y diseño, forma con propuesta de mejora, registro seguimiento pruebas y tiempo.
				\end{block}
			\end{frame}				
					
			\begin{frame}{Ciclo de Vida}
				\begin{figure}								
					\scalebox{0.5}{\includegraphics{Imagenes/Ciclo.png}} \caption{Ciclo de Vida PSP}
				\end{figure}			
			\end{frame}
			
			\begin{frame}{Requisitos}
				En está fase se definen los requisitos los cuales son definidos claramente en psp como:
					\begin{itemize}
						\item Descripción del problema.
						\item Especificación de componentes.
						\item Formas de procesos.
						\item Estimadores del tamaño del producto y tiempos en base a historicos.
					\end{itemize}
			\end{frame}					
			
			\begin{frame}{Ciclo de Vida}
				\begin{figure}								
					\scalebox{0.5}{\includegraphics{Imagenes/Ciclo.png}} \caption{Ciclo de Vida PSP}
				\end{figure}			
			\end{frame}						
			
\section{ISO/IEC 12207}
		\subsection{Introducci\'on}
		
			\begin{frame}
				\begin{center}
					\begin{block}{}
						\begin{center}
							{\huge ISO/IEC 12207}
						\end{center}
					\end{block}
				\end{center}
			\end{frame}					
		
			\begin{frame}{Introducci\'on ISO}
				La ISO (Organización Internacional de Normalización), es el organismo encargado de promover el desarrollo de normas internacionales de fabricación (tanto de productos como de servicios), comercio y comunicación para todas las ramas industriales a excepción de la eléctrica y la electrónica. Su función principal es la de buscar la estandarización de normas de productos y seguridad para las empresas u organizaciones (públicas o privadas) a nivel internacional.\\ \smallskip
			\end{frame}
		
			\begin{frame}{Introducción IEC}
				La IEC (Comisión Electrotécnica Internacional), es una organización de normalización en los campos eléctricos, eléctronicos, y tecnologías relacionadas.\\ \smallskip
				A la IEC se le debe el desarrollo y difusión de los estandares para algunas unidades de medida tales como Gauss, Hercio, Weber; Así como la propuesta de una sistema de unidades estándar. Numerosas Normas se desarrollan en conjunto con la ISO (normas ISO/IEC).
			\end{frame}
			
		\subsection{Historia}
			\begin{frame}{Historia}
				
			\end{frame}
			
		\subsection{Desarrollo1}
			\begin{frame}{Desarrollo1}
				contenido Desarrollo1
			\end{frame}

		\subsection{Desarrollo2}
			\begin{frame}{Desarrollo2}
				contenido Desarrollo2
			\end{frame}
			
\section{Conclusi\'on}
	\begin{frame}{Conclusión}
		aqu\'i va la conclusi\'on
	\end{frame}				
			
\section{Bibliograf\'ia}
	\begin{frame}{Bibliograf\'ia}
		\begin{thebibliography}{9}
		
			%item 1		
			\bibitem{mo02} 
 			Victor M. Fleites Sabido 
 			\newblock {\em Personal Software Process}, 
 			\newblock http://www.slideshare.net/Tonymx/introduccion-a-personal-software-process.
		
			%item 2		
			\bibitem{sm-hin04} 
 			Armando David Espinoza Robles
 			\newblock{\em Metodologías de Desarrollo de Software}, 
 			\newblock http://www.slideshare.net/juliopari/4-clase-metodologia-de-desarrolo-de-software.
 			
		http://calidadesoftware.wordpress.com/2012/02/23/personal-software-process/ 			
		http://es.pdfsb.com/readonline/5a56464364516835575846394358706755554d3d-118608
 			
 		http://ingsw.ccbas.uaa.mx/sitio/images/material/psp.htm
		\end{thebibliography}
	\end{frame}

\end{document}